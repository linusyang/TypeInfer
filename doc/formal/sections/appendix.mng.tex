\section{type check}

\subsection{Syntax}
\gram{
      \otte\ottinterrule
      \ottR\ottinterrule
      \ottt\ottinterrule
      %\ottG\ottinterrule
  }
\\[2.0mm]

\subsection{Operational Semantics: Call by name}

\newcommand{\castupe}{\ensuremath{\mathsf{cast}^{\uparrow}\ }}
\newcommand{\castdowne}{\ensuremath{\mathsf{cast}^{\downarrow}\ }}

\[
\inferrule*[right=S-Beta]
{  }
{ (\lambda x. e_1) e_2 \longrightarrow e_1 [x\mapsto e_2] }
\]

\[
\inferrule*[right=S-BetaAnn]
{  }
{ (\lambda x:\sigma. e_1) e_2 \longrightarrow e_1 [x\mapsto e_2] }
\]

\[
\inferrule*[right=S-App]
{ e_1 \longrightarrow e_1'  }
{ e_1 e_2 \longrightarrow e_1' e_2}
\]

\[
\inferrule*[right=S-CastDownUp]
{  }
{ \castdowne (\castupe e)  \longrightarrow e }
\]

\[
\inferrule*[right=S-CastDown]
{ e \longrightarrow e' }
{ \castdowne  e  \longrightarrow  \castdowne  e' }
\]

\[
\inferrule*[right=S-Ann]
{ e \longrightarrow e'  }
{ e:\sigma \longrightarrow e':\sigma}
\]

\[
\inferrule*[right=S-Let]
{  }
{ let\ x = e_1\ in\ e_2 \longrightarrow e_2[x\mapsto e_1] }
\]

\subsection{syntax-directed Typing}

\newcommand{\checktype}{\Gamma\vdash_\Downarrow}
\newcommand{\infertype}{\Gamma\vdash_\Uparrow}
\newcommand{\infercheck}{\Gamma\vdash_\delta}

\newcommand{\checktypeno}{\vdash_\Downarrow}
\newcommand{\infertypeno}{\vdash_\Uparrow}
\newcommand{\infercheckno}{\vdash_\delta}

\newcommand{\instinfer}{\vdash^{inst}_\Uparrow}
\newcommand{\instcheck}{\vdash^{inst}_\Downarrow}
\newcommand{\instinfercheck}{\vdash^{inst}_\delta}

\newcommand{\polyinfer}{\Gamma\vdash^{poly}_\Uparrow}
\newcommand{\polycheck}{\Gamma\vdash^{poly}_\Downarrow}
\newcommand{\polyinfercheck}{\vdash^{poly}_\delta}

\newcommand{\polymorphic}{\vdash^{dsk}}
\newcommand{\polymorphicstar}{\vdash^{dsk\star}}

\newcommand{\forallvars}[1]{\forall \overbar{#1}}


\framebox{$ \infercheck e : \rho$ } infer $\Uparrow$ check $\Downarrow$ $\delta = \Uparrow \mid \Downarrow$

\[
\hl{$
\inferrule*[right=AX]
{} {\infercheck \star : \star}
$}
\]

\[
\inferrule*[right=Var]
{x:\sigma \in \Gamma \\ \instinfercheck \sigma \sqsubseteq \rho } {\infercheck x : \rho}
\]

\[
\inferrule*[right=Lam-Infer]
{\Gamma,x: \tau \infertypeno e : \rho} {\infertype (\lambda x.\, e) : (\Pi x: \tau. \rho)}
\]

\[
\inferrule*[right=Lam-Check]
{\Gamma,x: \sigma_1 \polycheck e : \sigma_2 } {\checktype (\lambda x.\, e) : (\Pi x: \sigma_1. \sigma_2)}
\]

\[
\inferrule*[right=LamAnn-Infer]
{
\hl{$\checktype \sigma:\star$} \\
\Gamma,x: \sigma \infertypeno e : \rho } {\infertype (\lambda x:\sigma.\, e) : (\Pi x: \sigma. \rho)}
\]

\[
\inferrule*[right=LamAnn-Check]
{
\hl{$\checktype \sigma:\star$} \\
\polymorphic \sigma_1 \sqsubseteq \sigma \\ \Gamma,x: \sigma \polycheck e : \sigma_2 } {\checktype (\lambda x:\sigma.\, e) : (\Pi x: \sigma_1. \sigma_2)}
\]

\[
\inferrule*[right=App]
{\infertype e_1 : (\Pi x:\sigma_1. \sigma_2) \\
\polycheck e_2 : \sigma_1 \\
\instinfercheck \hl{$\sigma_2 [x \mapsto e_2]$} \sqsubseteq \rho}
{\infercheck e_1\,e_2 : \rho}
\]

\[
\hl{$
\inferrule*[right=ExplicitPi]
{\checktype \tau_1 : \star \\ \Gamma, x:\tau_1 \checktypeno \tau_2 : \star} {\infercheck (\Pi x:\tau_1. \tau_2) : \star}
$}
\]

\[
\inferrule*[right=Ann]
{
\hl{$\checktype \sigma : \star$} \\
\polycheck (e:\sigma) \\
\instinfercheck \sigma : \rho }
{\infercheck (e : \sigma) : \rho }
\]

\[
\hl{$
\inferrule*[right=CastUp-Check]
{ \rho \longrightarrow \sigma \\ \polycheck e : \sigma } {\checktype (\castupe e) : \rho}
$}
\]

\[
\hl{$
\inferrule*[right=CastDown]
{\infertype e : \rho_1 \\ \rho_1 \longrightarrow \sigma \\ \instinfercheck \sigma \sqsubseteq \rho_2} {\infercheck (\castdowne e) : \rho_2}
$}
\]

\[
\inferrule*[right=Let]
{\polyinfer e_1 : \sigma \\
\hl{$\infercheck e_2[x \mapsto e_1]: \rho$}}
{\infercheck ( let\ x = e_1\ in\ e_2 ) : \rho }
\]

\framebox{$ \infercheck \sigma : \star$ }

\[
\hl{$
\inferrule*[right=ImplicitPi]
{\checktype \tau : \star \\ \Gamma, x:\tau \checktypeno \rho : \star} {\infercheck (\forall x : \tau. \rho) : \star}
$}
\]

\framebox{$ \infercheck \rho : \star$ }

\[
\hl{$
\inferrule*[right=FunPoly]
{\checktype \sigma_1 : \star \\ \Gamma, x:\sigma_1 \checktypeno \sigma_2 : \star} {\infercheck (\Pi x : \sigma_1. \sigma_2) : \star}
$}
\]

\framebox{$ \Gamma \polyinfercheck e : \sigma$ }

\[
\inferrule*[right=Gen-Infer]
{\infertype e :\rho \\ \overbar{\hl{$x:\tau$}}=ftv(\rho)-ftv(\Gamma) \\
\hl{$\checktype (\forallvars{x:\tau}. \rho):\star$} } {\polyinfer e :(\forallvars{x:\tau}. \rho)}
\]

\[
\inferrule*[right=Gen-Check]
{
pr(\sigma) = \forallvars{\hl{$x:\tau$}}. \rho \\
\overbar{\hl{$x:\tau$}} \notin ftv(\Gamma) \\
\checktype e :\rho
} {\polycheck e : \sigma}
\]

\framebox{$ \instinfercheck \sigma \sqsubseteq \rho$ }

\[
\inferrule*[right=Inst-Infer]
{\hl{$\overbar{\tau_\beta : \tau}$}}
{\instinfer \forallvars{x:\tau}. \rho \sqsubseteq \rho[\overbar{x \mapsto \tau_\beta}]}
\]

\[
\inferrule*[right=Inst-Check]
{\polymorphic \sigma \sqsubseteq \rho } {\instcheck \sigma \sqsubseteq \rho}
\]

\framebox{$pr(\sigma)=\forallvars{x:\tau}.\rho$: float out all $\forall$s, the same as the paper}

\[
\inferrule*[right=PR-POLY]
{pr(\rho_1) = \forallvars{b:\tau_2}. \rho_2 \\ \overbar{a} \cap \overbar{b} = \emptyset} {pr(\forallvars{a:\tau_1}. \rho_1) = \forallvars{a:\tau_1. b:\tau_2}. \rho_2}
\]

\[
\inferrule*[right=PR-FUN]
{pr(\sigma_2) = \forallvars{a:\tau}. \rho_2 \\ \overbar{a} \cap ftv(\sigma_1) = \emptyset} {pr(\Pi x:\sigma_1. \sigma_2) = \forallvars{a:\tau}. \Pi x:\sigma_1. \rho_2}
\]

\[
\inferrule*[right=\hl{PR-OTHER-CASE}]
{  } {pr(\tau)=\tau}
\]

\framebox{$\Sigma \polymorphic \sigma_1 \sqsubseteq \sigma_2$: almost same, take care of kinds}

\[
\inferrule*[right=DEEP-SKOL]
{pr(\sigma_2)=\forallvars{a:\tau}. \rho \\ a \notin ftv(\sigma_1) \\
\hl{$\Sigma \vdash \overbar{a:\tau}$} \\
\hl{$\Sigma, \overbar{a:\tau}$} \polymorphicstar \sigma_1 \sqsubseteq \rho }
{\Sigma \polymorphic \sigma_1 \sqsubseteq \sigma_2}
\]

\framebox{$\Sigma \polymorphicstar \sigma_1 \sqsubseteq \rho$}

\[
\hl{$
\inferrule*[right=Alpha-Equal]
{  \alpha-equal(\sigma_1, \sigma_2)  }
{ \Sigma \polymorphicstar \sigma_1 \sqsubseteq \sigma_2 }
$}
\]

\[
\inferrule*[right=SPEC]
{\hl{$\Sigma \vdash \overbar{\beta: \tau}$} \\
\hl{$\Sigma, \overbar{\beta:\tau}$} \polymorphicstar \rho_1[\overbar{a \mapsto \beta}] \sqsubseteq \rho_2}
{\Sigma \polymorphicstar \forallvars{a:\tau}.\rho_1 \sqsubseteq \rho_2}
\]

\[
\inferrule*[right=FUN]
{\Sigma \polymorphic  \sigma_3 \sqsubseteq \sigma_1 \\
\hl{$ \Sigma \vdash x:\sigma_1$} \\
\hl{$ \Sigma, x:\sigma_1$} \polymorphicstar  \sigma_2 \sqsubseteq \rho_4}
{\hl{$\Sigma$} \polymorphicstar \Pi x:\sigma_1. \sigma_2 \sqsubseteq \Pi x:\sigma_3. \rho_4}
\]

\[
\hl{$
\inferrule*[right=APP]
{
\tau_1 \sigma_1 \longrightarrow \sigma_2 \\
\tau_2 \sigma_2 \longrightarrow \sigma_3 \\
\Sigma \polymorphic  \sigma_2 \sqsubseteq \sigma_3 }
{\Sigma \polymorphicstar \tau_1\, \sigma_1 \sqsubseteq \tau_2\, \sigma_1 }
$}
\]

\[
\hl{$
\inferrule*[right=ANN]
{
\Sigma \polymorphic  \sigma_1 \sqsubseteq \sigma_3 }
{\Sigma \polymorphicstar \sigma_1:\sigma_2 \sqsubseteq \sigma_3:\sigma_2 }
$}
\]

\[
\hl{$
\inferrule*[right=CASTDOWN]
{
\castdowne \sigma_1 \longrightarrow \sigma_3 \\
\castdowne \sigma_2 \longrightarrow \sigma_4 \\
\Sigma \polymorphic  \sigma_3 \sqsubseteq \sigma_4 }
{\Sigma \polymorphicstar  (\castdowne \sigma_1) \sqsubseteq  (\castdowne \sigma_2)}
$}
\]

\[
\hl{$
\inferrule*[right=LET]
{
(let\ x = e_1\ in\ \sigma_1) \longrightarrow \sigma_3 \\
(let\ x = e_1\ in\ \sigma_2) \longrightarrow \sigma_4 \\
\Sigma \polymorphic  \sigma_3 \sqsubseteq \sigma_4 }
{\Sigma \polymorphicstar  (let\ x = e_1\ in\ \sigma_1) \sqsubseteq  (let\ x = e_1\ in\ \sigma_2) }
$}
\]

\framebox{$\Sigma \vdash a: \tau$: two variable with same name should have same kind}

\[
\hl{$
\inferrule*[right=EXISTS]
{ a:\tau \in \Sigma}
{ \Sigma \vdash a:\tau}
$}
\]

\[
\hl{$
\inferrule*[right=NOT\_EXISTS]
{ b \notin \Sigma}
{ \Sigma \vdash b:\tau}
$}
\]

\clearpage

\subsection{Unification}

\begin{algorithm}
\caption{Unification}\label{euclid}
\begin{algorithmic}[1]
\algloopdefx{Let}[1]{\textbf{Let} #1 \textbf{In}}
\algloop[In]{In}
\Function{Unify(C)}{}
\If {C = $\emptyset$}
    \Return []
\Else
    \Let {$\{S=T\} \cup C' = C$}
        \If{$\alpha$-equal S T}
            \State unify($C'$)
        \ElsIf{$S =$ TVar $X$ $K$ and $X \notin FV(T)$ and $T\in \tau$}
            \State $C''$ = checktype $T$ $K$
            \State unify($[X\mapsto T](C \cup C'')) \circ [X\mapsto T]$
        \ElsIf{$T =$ TVar $X$ $K$ and $X \notin FV(S)$ and $S\in \tau$}
            \State $C''$ = checktype $S$ $K$
            \State unify($[X\mapsto S](C \cup C'')) \circ [X\mapsto S]$
        \ElsIf{$S = S_1 \rightarrow S_2$ and $T = T_1 \rightarrow T_2$}
            \State unify($C' \cup \{S_1=T_1, S_2=T_2\}$)
        \ElsIf{$S = cast^\uparrow S_1$ and $T = cast^\uparrow T_1$}
            \State unify($C' \cup \{S_1=T_1\}$)
        \ElsIf{$S = cast^\downarrow S_1$ and $T = cast^\downarrow T_1$}
            \State unify($C' \cup \{S_1=T_1\}$)
        \ElsIf{$S =$ App $S_1$ $S_2$ and $T =$ App $T_1$ $T_2$}
            \State unify($C' \cup \{S_1=T_1, S_2=T_2\}$)
        \ElsIf{$S =$ Ann $S_1$ $S_2$ and $T =$ Ann $T_1$ $T_2$}
            \State unify($C' \cup \{S_1=T_1, S_2=T_2\}$)
        \ElsIf{$S =\lambda x. S_1$  and $T =\lambda x. T_1$}
            \State unify($C' \cup \{S_1=T_1\}$)
        \ElsIf{$S =\lambda x:S_1. S_2$  and $T =\lambda x:T_1. T_2$}
            \State unify($C' \cup \{S_1=T_1, S_2=T_2\}$)
        \Else
            \State fail
        \EndIf
\EndIf
\EndFunction
\end{algorithmic}
\end{algorithm}

\clearpage

\subsection{Translation}

\newcommand{\transto}[1]{\hl{$\leadsto#1$}}
\newcommand{\translate}[2]{\colorbox{cyan!30}{$\checktype#1:\star\leadsto#2$}}
\newcommand{\cyancolorbox}[1]{\colorbox{cyan!30}{$#1$}}
\newcommand{\invariant}[2]{\leadsto #1 :#2}

\framebox{$ \infercheck e : \rho \invariant{t}{|\rho|}$ } infer $\Uparrow$ check $\Downarrow$ $\delta = \Uparrow \mid \Downarrow$

\[
\inferrule*[right=AX]
{} {\infercheck \star : \star \transto{\star}}
\]

\[
\inferrule*[right=Var]
{x:\sigma \in \Gamma \\ \instinfercheck \sigma \sqsubseteq \rho \transto{f} } {\infercheck x : \rho \transto{f\ x}}
\]

\[
\inferrule*[right=Lam-Infer]
{\Gamma,x: \tau \infertypeno e : \rho \transto{t} \\
\translate{\tau}{\tau'}} {\infertype (\lambda x.\, e) : (\Pi x: \tau. \rho) \transto {\lambda(x:\tau'). t}}
\]

\[
\inferrule*[right=Lam-Check]
{\Gamma,x: \sigma_1 \polycheck e : \sigma_2 \transto{t}\\
\translate{\sigma_1}{\sigma_1'}
} {\checktype (\lambda x.\, e) : (\Pi x: \sigma_1. \sigma_2) \transto{\lambda(x:\sigma_1').t}}
\]

\[
\inferrule*[right=LamAnn-Infer]
{
\checktype \sigma:\star \\
\Gamma,x: \sigma \infertypeno e : \rho \transto{t} \\
\translate{\sigma}{\sigma'}
} {\infertype (\lambda x:\sigma.\, e) : (\Pi x: \sigma. \rho) \transto{\lambda(x:\sigma'). t}}
\]

\[
\inferrule*[right=LamAnn-Check]
{
\checktype \sigma:\star \\
\polymorphic \sigma_1 \sqsubseteq \sigma \transto{f} \\
\Gamma,x: \sigma \polycheck e : \sigma_2 \transto{t} \\
\translate{\sigma_1}{\sigma_1'}
}
{\checktype (\lambda x:\sigma.\, e) : (\Pi x: \sigma_1. \sigma_2) \transto{ \lambda(x:\sigma_1'). t[x\mapsto f\ x] }}
\]

\[
\inferrule*[right=App]
{\infertype e_1 : (\Pi x:\sigma_1. \sigma_2) \transto {t_1} \\
\polycheck e_2 : \sigma_1 \transto{t_2}\\
\instinfercheck \sigma_2 [x \mapsto e_2] \sqsubseteq \rho \transto{f}}
{\infercheck e_1\,e_2 : \rho \transto{f\ (t_1\ t_2)}}
\]

\[
\inferrule*[right=ExplicitPi]
{\checktype \tau_1 : \star \transto{\tau_1'} \\ \Gamma, x:\tau_1 \checktypeno \tau_2 : \star \transto{\tau_2'}}
{\infercheck (\Pi x:\tau_1. \tau_2) : \star \transto{(\Pi x:\tau_1'. \tau_2')}}
\]

\[
\inferrule*[right=Ann]
{
\checktype \sigma : \star \\
\polycheck (e:\sigma) \transto{t}\\
\instinfercheck \sigma : \rho \transto{f}}
{\infercheck (e : \sigma) : \rho \transto{f\ t}}
\]

\[
\inferrule*[right=CastUp-Check]
{ \rho \longrightarrow \sigma \\ \polycheck e : \sigma \transto{t} \\
\translate{\rho}{\rho'}} {\checktype (\castupe e) : \rho \transto{\castupe [\rho']\ t}}
\]

\[
\inferrule*[right=CastDown]
{\infertype e : \rho_1 \transto{t}\\
\rho_1 \longrightarrow \sigma \\
\instinfercheck \sigma \sqsubseteq \rho_2\transto {f}}
{\infercheck (\castdowne e) : \rho_2 \transto {f\ (\castdowne t)}}
\]

\[
\inferrule*[right=Let]
{\polyinfer e_1 : \sigma \transto{t_1}\\
\infercheck e_2[x \mapsto e_1]: \rho \transto{t_2}}
{\infercheck ( let\ x = e_1\ in\ e_2 ) : \rho \transto{t_2}}
\]

\framebox{$ \infercheck \sigma : \star \invariant{t}{*}$ }

\[
\inferrule*[right=ImplicitPi]
{\checktype \tau : \star \transto{\tau'}\\ \Gamma, x:\tau \checktypeno \rho : \star \transto{\rho'}}
{\infercheck (\forall x : \tau. \rho) : \star \transto {\Pi (x:\tau'). \rho'}}
\]

\framebox{$ \infercheck \rho : \star \invariant{t}{*}$ }

\[
\inferrule*[right=FunPoly]
{\checktype \sigma_1 : \star \transto{\sigma_1'}\\
\Gamma, x:\sigma_1 \checktypeno \sigma_2 : \star \transto{\sigma_2'}}
{\infercheck (\Pi x : \sigma_1. \sigma_2) : \star \transto{\Pi (x:\sigma_1'). \sigma_2'}}
\]

\framebox{$ \Gamma \polyinfercheck e : \sigma \invariant{t}{|\sigma|}$ }

\[
\inferrule*[right=Gen-Infer]
{\infertype e :\rho \transto{t} \\ \overbar{x:\tau}\transto{\overbar{\tau_1}}=ftv(\rho)-ftv(\Gamma) \\
\checktype (\forallvars{x:\tau}. \rho):\star } {\polyinfer e :(\forallvars{x:\tau}. \rho) \transto{\lambda (\overbar{x:\tau_1}). t}}
\]

\[
\inferrule*[right=Gen-Check]
{
pr(\sigma) = \forallvars{x:\tau}. \rho \transto{f} \\
\overbar{x:\tau} \transto{\overbar{\tau_1}}\notin ftv(\Gamma) \\
\checktype e :\rho \transto{t}
} {\polycheck e : \sigma \transto{f\ (\lambda (\overbar{x:\tau_1}). t)}}
\]

\framebox{$ \instinfercheck \sigma \sqsubseteq \rho \invariant{f}{|\sigma|\to|\rho|}$ }

\[
\inferrule*[right=Inst-Infer]
{\overbar{\tau_\beta \transto {\tau_{\beta 1}}: \tau} \\
\translate{\forallvars{x:\tau} . \rho}{\sigma_1}
}
{\instinfer \forallvars{x:\tau} . \rho \sqsubseteq \rho[\overbar{x \mapsto \tau_\beta}] \transto {\lambda (a: \sigma_1). (a\ \overbar{\tau_{\beta 1}})}}
\]

\[
\inferrule*[right=Inst-Check]
{\polymorphic \sigma \sqsubseteq \rho \transto {f}
}
{\instcheck \sigma \sqsubseteq \rho \transto {f}}
\]

\framebox{$pr(\sigma)=\forallvars{x:\tau}.\rho \invariant{f}{|\forallvars{x:\tau}.\rho|\to|\sigma|}$}

\[
\inferrule*[right=PR-POLY]
{pr(\rho_1) = \forallvars{b:\tau_2}. \rho_2 \transto{f}\\ \overbar{a} \cap \overbar{b} = \emptyset \\
\translate{\forallvars{a:\tau_1. b:\tau_2}. \rho_2  }{ \sigma} \\
\translate{\tau_1}{\tau_3}
}
{pr(\forallvars{a:\tau_1}. \rho_1) = \forallvars{a:\tau_1. b:\tau_2}. \rho_2 \\
\transto{\lambda (x:\sigma).\lambda (\overbar{a:\tau_3}). f\ (x\ \overbar{a})}}
\]

\[
\inferrule*[right=PR-FUN]
{pr(\sigma_2) = \forallvars{a:\tau}. \rho_2 \transto{f} \\ \overbar{a} \cap ftv(\sigma_1) = \emptyset\\
\translate{\Pi (\overbar{a:\tau}) (\overbar{b:\sigma_1}). \rho_2}{ \sigma_3 } \\
\translate{\sigma_1}{\sigma_4} \\
\translate{\tau}{\tau_2}
}
{pr(\Pi x:\sigma_1. \sigma_2) = \forallvars{a:\tau}. \Pi x:\sigma_1. \rho_2 \\
\transto{\lambda (x:\sigma_3). \lambda (y:\sigma_4). f\ (\lambda (\overbar{a:\tau_2}). x\ a\ y) }}
\]

\[
\inferrule*[right=PR-OTHER-CASE]
{ \translate{\tau}{\tau_2} } {pr(\tau)=\tau \transto {\lambda (x:\tau_2). x}}
\]

\framebox{$\Sigma \polymorphic \sigma_1 \sqsubseteq \sigma_2 \invariant{f}{|\sigma_1|\to|\sigma_2|}$}

\[
\inferrule*[right=DEEP-SKOL]
{pr(\sigma_2)=\forallvars{a:\tau}. \rho \transto{f_1}\\ a \notin ftv(\sigma_1) \\
\Sigma \vdash \overbar{a:\tau} \transto {\tau_1} \\
\Sigma, \overbar{a:\tau} \polymorphicstar \sigma_1 \sqsubseteq \rho \transto{f_2}}
{\Sigma \polymorphic \sigma_1 \sqsubseteq \sigma_2 \transto{\lambda x:\sigma_1. f_1\ (\lambda(\overbar{a:\tau_1}). f_2\ x) }}
\]

\framebox{$\Sigma \polymorphicstar \sigma \sqsubseteq \rho \invariant{f}{|\sigma| \to|\rho|}$}

\[
\inferrule*[right=Alpha-Equal]
{  \alpha-equal(\sigma_1, \sigma_2)  }
{ \Sigma \polymorphicstar \sigma_1 \sqsubseteq \sigma_2 \transto{\lambda x:*. x}}
\]

\[
\inferrule*[right=SPEC]
{\Sigma \vdash \overbar{\beta: \tau} \transto{\overbar{\beta_1}} \\
\Sigma, \overbar{\beta:\tau} \polymorphicstar \rho_1[\overbar{a \mapsto \beta}] \sqsubseteq \rho_2 \transto{f}
}
{\Sigma \polymorphicstar \forallvars{a:\tau}.\rho_1 \sqsubseteq \rho_2 \\
\transto {\lambda x:( \Pi \overbar{a:\tau}.\rho_1  ). f\ (x\ \overbar{\beta_1}) }}
\]

\[
\inferrule*[right=FUN]
{\Sigma \polymorphic  \sigma_3 \sqsubseteq \sigma_1 \transto{f_1} \\
 \Sigma \vdash x:\sigma_1 \\
 \Sigma, x:\sigma_1 \polymorphicstar  \sigma_2 \sqsubseteq \rho_4 \transto{f_2} \\
 \translate{\sigma_3}{\sigma_5}
 }
{\Sigma \polymorphicstar \Pi x:\sigma_1. \sigma_2 \sqsubseteq \Pi x:\sigma_3. \rho_4 \\
\transto {\lambda x: ( \Pi x:\sigma_1. \sigma_2  ).\lambda y:\sigma_5 . f_2\ (x\ (f_1\ y))}
}
\]

\[
\inferrule*[right=APP]
{
\tau_1 \sigma_1 \longrightarrow \sigma_2 \\
\tau_2 \sigma_2 \longrightarrow \sigma_3 \\
\Sigma \polymorphic  \sigma_2 \sqsubseteq \sigma_3 \transto {f}}
{\Sigma \polymorphicstar \tau_1\, \sigma_1 \sqsubseteq \tau_2\, \sigma_1 \\
\transto {\lambda x:\tau_1 \sigma_1. \castupe [\tau_2 \sigma_1]\ f\ (\castdowne x)}}
\]

\[
\inferrule*[right=ANN]
{
\Sigma \polymorphic  \sigma_1 \sqsubseteq \sigma_3 \transto{f}}
{\Sigma \polymorphicstar \sigma_1:\sigma_2 \sqsubseteq \sigma_3:\sigma_2 \transto {f}}
\]

\[
\inferrule*[right=CASTDOWN]
{
\castdowne \sigma_1 \longrightarrow \sigma_3 \\
\castdowne \sigma_2 \longrightarrow \sigma_4 \\
\Sigma \polymorphic  \sigma_3 \sqsubseteq \sigma_4 \transto {f}}
{\Sigma \polymorphicstar  (\castdowne \sigma_1) \sqsubseteq  (\castdowne \sigma_2)
\transto{\lambda x:(\castdowne \sigma_1). \castupe [\castdowne \sigma_2]\ f\ (\castdowne x)}}
\]

\[
\inferrule*[right=LET]
{
(let\ x = e_1\ in\ \sigma_1) \longrightarrow \sigma_3 \\
(let\ x = e_1\ in\ \sigma_2) \longrightarrow \sigma_4 \\
\Sigma \polymorphic  \sigma_3 \sqsubseteq \sigma_4 \transto {f}}
{\Sigma \polymorphicstar  (let\ x = e_1\ in\ \sigma_1) \sqsubseteq  (let\ x = e_1\ in\ \sigma_2) \\
\transto {\lambda x:(let\ x = e_1\ in\ \sigma_1). \castupe [(let\ x = e_1\ in\ \sigma_2)]\ f\ (\castdowne x)}}
\]

\framebox{$\Sigma \vdash a: \tau$: two variable with same name should have same kind}

\[
\inferrule*[right=EXISTS]
{ a:\tau \in \Sigma}
{ \Sigma \vdash a:\tau}
\]

\[
\inferrule*[right=NOT\_EXISTS]
{ b \notin \Sigma}
{ \Sigma \vdash b:\tau}
\]
